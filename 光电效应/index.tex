\documentclass{article}
\usepackage{booktabs}
\usepackage{multirow}
\usepackage{tikz}
\usepackage{pgfplots}

% I'm using MaxTeX on Catalina, the CTeX-kit not work well now(2019.11.24).
% So I'm using luatexja and fontspec to specify the CJK font.
% You may not have Adobe Std, change it to your font
% Or delete following two lines and use ctexart on Linux/Windows.
% I will change it when the CTeX-kit is fixed.
% see: https://github.com/CTeX-org/ctex-kit/pull/462
\usepackage{luatexja-fontspec}
\setmainjfont{Adobe Kaiti Std}
\begin{document}

\begin{table}[]
\centering
\caption{测量普朗克常数$h$及红限频率数据表}
\label{tab:table1}
\begin{tabular}{@{}c|ccccc@{}}
\toprule
波长\lambda /nm & 365    & 405    & 436    & 546    & 577    \\ \toprule
\multirow{4}{*}{$U_s$/V}       & -1.458 & -1.300 & -1.003 & -0.434 & -0.313 \\
                             & -1.472 & -1.321 & -1.035 & -0.456 & -0.336 \\
                             & -1.495 & -1.326 & -10.43 & -0.468 & -0.346 \\
                             & -1.502 & -1.335 & -1.052 & -0.477 & -0.365 \\ \midrule
平均值/V                         & -1.482 & -1.321 & -1.033 & -0.459 & -0.338 \\ \bottomrule
\end{tabular}
\end{table}

\begin{table}[]
\centering
\caption{测量光电管伏安特性数据表}
\label{tab:table2}
\begin{tabular}{@{}c|cc|cc|cc@{}}
\toprule
光阑/mm                & \multicolumn{2}{c|}{2} & \multicolumn{2}{c|}{4} & \multicolumn{2}{c}{8} \\ \midrule
测量量                  & U/V   & $I/10^{-13}A$ & U/V   & $I/10^{-12}A$  & U/V   & $I/10^{-12}A$  \\ \midrule
\multirow{13}{*}{数据} & -0.4  & 0.00          & -0.3  & 0.00           & -0.3  & 0.00           \\
                     & 1.1   & 577           & 1.2   & 180.2          & 1.2   & 383            \\
                     & 2.6   & 908           & 2.7   & 336            & 2.7   & 779            \\
                     & 4.1   & 1114          & 4.2   & 408            & 4.2   & 1129           \\
                     & 5.6   & 1293          & 5.7   & 479            & 5.7   & 1406           \\
                     & 7.1   & 1460          & 7.2   & 543            & 7.2   & 1623           \\
                     & 8.6   & 1620          & 8.7   & 602            & 8.7   & 1795           \\
                     & 10.1  & 1768          & 10.2  & 657            & 10.2  & 1948           \\
                     & 11.6  & 1889          & 11.7  & 707            & 11.7  & 2090           \\
                     & 13.1  & 1997          & 13.2  & 758            & 13.2  & 2270           \\
                     & 14.6  & 2150          & 14.7  & 799            & 14.7  & 2390           \\
                     & 16.1  & 2230          & 16.2  & 835            & 16.2  & 2520           \\
                     & 17.6  & 2310          & 17.7  & 863            & 17.7  & 2640           \\
                     & 19.1  & 2370          & 19.2  & 906            & 19.2  & 2720           \\
                     & 20.6  & 2470          & 20.7  & 932            & 20.7  & 2810           \\
                     & 22.1  & 2530          & 22.2  & 958            & 22.2  & 2870           \\
                     & 23.6  & 2580          & 23.7  & 981            & 23.7  & 2940           \\
                     & 25.1  & 2660          & 25.2  & 998            & 25.2  & 3030           \\
                     & 26.6  & 2700          & 26.7  & 1021           & 26.7  & 3080           \\
                     & 28.1  & 2750          & 28.2  & 1050           & 28.2  & 3170           \\
                     & 29.6  & 2800          & 29.7  & 1062           & 29.7  & 3230           \\ \bottomrule
\end{tabular}
\end{table}

\begin{table}[H]
\centering
\caption{验证光饱和电流与入射光强正比关系数据表}
\label{tab:table3}
\begin{tabular}{@{}c|c|ccc@{}}
\toprule
波长/nm                & \multicolumn{4}{c}{数据}           \\ \midrule
\multirow{2}{*}{546} & 光阑/mm       & 2     & 4    & 8    \\
                     & $I/10^{-12}A$ & 174.3 & 669  & 2660 \\ \midrule
\multirow{2}{*}{436} & 光阑/mm       & 2     & 4    & 8    \\
                     & $I/10^{-12}A$ & 514   & 1879 & 6860 \\ \bottomrule
\end{tabular}
\end{table}

\begin{tikzpicture}
    \begin{axis}[
        legend style={at={(0.5,-0.2)},
		anchor=north,legend columns=-1},
        xlabel=$U_s$/V,
        ylabel=$I/10^{-12}$/A]
 \addplot [mark = *, blue, smooth, very thick] table {1.csv};
    \addlegendentry{光阑2mm}
 \addplot [mark = *, red, smooth, very thick] table {2.csv};
    \addlegendentry{光阑4mm}
 \addplot [mark = * , green, smooth, very thick] table {3.csv};
    \addlegendentry{光阑8mm}
    \end{axis}
\draw[style=help lines] (0,0) grid (6.9,5.7)
\end{tikzpicture}
\\
\begin{tikzpicture}
    \begin{axis}[
        legend style={at={(0.3,0.8)},
		anchor=south,legend columns=-1},
        xlabel=\Phi/$mm^2$,
        ylabel=$I/10^{-11}$A]
        \addplot [mark = *, blue, smooth, very thick] table {4.csv};
    \addlegendentry{546nm}
        \addplot [mark = *, red, smooth, very thick] table {5.csv};
    \addlegendentry{436nm}
    \end{axis} 

\end{tikzpicture}
\end{document}
